\newpage
\storeareas\riskvalues
\KOMAoptions{paper=a3, paper=landscape, DIV=current}
\areaset
  {\dimexpr\the\paperwidth-2cm\relax}% calculate requiered \textwidth
  {\dimexpr\the\paperheight-5.5cm\relax}% calculate requiered \textheight
\recalctypearea

\chapter{Risikoanalyse}
\begin{table}[H]
    \begin{tabular}{ |C{0.01\textwidth}|C{0.1\textwidth}|C{0.1\textwidth}|C{0.02\textwidth}|C{0.02\textwidth}|C{0.1\textwidth}|C{0.1\textwidth}|C{0.05\textwidth}|C{0.05\textwidth}|C{0.1\textwidth}|C{0.1\textwidth}|C{0.1\textwidth}|C{0.2\textwidth}| }
        \hline
        Nr & Riskiobeschreibung & Auswirkung & W & S & Risiko & Handlungsweise & Massnahme &  W & S & Risiko & Handlungsweise \\
        \hline 
        1 & Krankheit & Weniger Arbeitstage & W3 & S2-4 & \cellcolor{red}Hoch & Keine Risken eingehen bei Kälte, immer genug Kleidung tragen 
        & W3 & S3 & \cellcolor{yellow}Mittel & Risikoakzeptanz &\\
        \hline
        2 & Datenverslust &Datenverlus und Zeitverzug & W2 & S4 & \cellcolor{red}Hoch &Dateisicherungskonzept erstellen und dies strikt einhalten 
        & W2 & S2 & \cellcolor{green}Niedrig & Risikoakzeptanz &\\
        \hline
        3 & Ausfall des Laptops & Zeistverzug & W3 & S2 & \cellcolor{yellow}Niedrig &Sorge tragen und ein backup Gerät bereitstellen 
        & W1 & S1 & \cellcolor{green}Niedrig & Risikoakzeptanz &\\
        \hline
    \end{tabular}
    \caption{Risikoanalyse}
\end{table}

Schadensausmaß: \\
S1 = führt zu keiner Abwertung \\
S2 = geringe Abwertung bis 1.0 Notenpunkte \\
S3 = hohe Abwertung über 1,0 Notenpunkte \\
S4 = führt zu Nichtbestehen \\
\newline
Eintrittswahrscheinlichkeit: \\
W1 = unvorstellbar \\
W2 = unwahrscheinlich \\
W3 = eher vorstellbar \\
W4 = vorstellbar \\
W5= Eintreffen hoch \\

\restoregeometry
\riskvalues
\newpage