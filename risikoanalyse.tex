\newpage
\storeareas\riskvalues
\KOMAoptions{paper=a3, paper=landscape, DIV=current}
\areaset
  {\dimexpr\the\paperwidth-1cm\relax}% calculate requiered \textwidth
  {\dimexpr\the\paperheight-5.5cm\relax}% calculate requiered \textheight
\recalctypearea

\chapter{Risikoanalyse}
\begin{table}[H]
    \begin{tabular}{ |C{0.01\textwidth}|C{0.1\textwidth}|C{0.1\textwidth}|C{0.02\textwidth}|C{0.02\textwidth}|C{0.03\textwidth}|C{0.1\textwidth}|C{0.2\textwidth}|C{0.02\textwidth}|C{0.02\textwidth}|C{0.03\textwidth}|C{0.1\textwidth}| }
        \hline
        \multirow{2}*{Nr} & \multirow{2}*{Riskiobeschreibung} & \multirow{2}*{Auswirkung} & \multicolumn{4}{|l|}{Vor Massnahme}& \multirow{2}*{Massnahmen} & \multicolumn{4}{|l|}{Nach Massnahme} \\
        \cline{4-7} \cline{9-12}&&& W & S & Risiko & Handlungsweise &&  W & S & Risiko & Handlungsweise \\
        \hline 
        1 & Krankheit & Weniger Arbeitstage & W5 & S4 & \cellcolor{red}Hoch & Risikominderung &       
        \begin{itemize}[noitemsep,topsep=0pt]
        \item Die Empfehlungen vom BAG beachten
          \item Die Arbeit im Homeoffice erledigen
          \item Ansammlungen von Leuten vermeiden
          \item Experten sofort informieren
        \end{itemize} & W5 & S1 & \cellcolor{yellow}Mittel & Risikoakzeptanz \\
        \hline
        2 & Datenverslust & Relevante Daten der Arbeit gehen verloren. & W1 & S3 & \cellcolor{green}Niedrig &Risikominderung 
        & Dateisicherungskonzept erstellen und dies strikt einhalten & W1 & S1 & \cellcolor{green}Niedrig & Risikoakzeptanz \\
        \hline
        3 & Ausfall des Laptops & Zeistverzug & W3 & S2 & \cellcolor{yellow}Mittel & Risikominderung 
        & Sorge tragen und ein backup Gerät bereitstellen & W2 & S2 & \cellcolor{green}Niedrig & Risikoakzeptanz\\
        \hline
        4 & Abbruch wegen Pandemie & Verschiebung der IPA & W2 & S4 & \cellcolor{yellow}Mittel & Risikoakzeptanz  & Aufgrund der Skala des Risikos kann ich als Teilnehmer nichts dagegen machen. Risikoakzeptanz & W2 & S4 & \cellcolor{yellow}Mittel & Risikoakzeptanz\\
        \hline
    \end{tabular}
    \caption{Risikoanalyse}
\end{table}

Schadensausmass: \\
S1 = führt zu keiner Abwertung \\
S2 = geringe Abwertung bis 1.0 Notenpunkte \\
S3 = hohe Abwertung über 1,0 Notenpunkte \\
S4 = führt zu Nichtbestehen \\
\newline
Eintrittswahrscheinlichkeit: \\
W1 = unvorstellbar \\
W2 = unwahrscheinlich \\
W3 = eher vorstellbar \\
W4 = vorstellbar \\
W5= Eintreffen hoch \\

\restoregeometry
\riskvalues
\newpage