\part{Ablauf Organisation und Umfeld}
\chapter{Kurzfassung des IPA-Berichtes}
\section{Ausgangssituation}
\section{Umsetzung}
\section{Ergebnis}

\tableofcontents



\chapter{Aufgabenstellung}
\section{Titel der Arbeit}
\section{Thematik}
\section{Ausgangslage}
\section{Detaillierte Aufgabenstellung}
\section{Mittel und Methoden inklusive Projektmethode}
\section{Vorkenntnisse}
\section{Vorarbeiten}
\section{Neue Lerninhalte}
\section{Arbeiten in den letzten 6 Monaten}


\chapter{Standards}

\chapter{Schutzwertanalyse}
%Dateisicherung nicht mehr verlangt

\chapter{Organisation der IPA}
\section{Datensicherung der IPA}
\section{Änderungskontrolle, Prüfung, Genehmigung}
\begin{table}[h!]
    \begin{tabular}{|L{0.1\textwidth}|L{0.23\textwidth}|L{0.23\textwidth}|L{0.33\textwidth}|}
        \hline
        \rowcolor{puzzleblue}Version & Datum & Name & Beschreibung \\[11pt]
        \hline
        Vorlage & 07.11.2019 & Sylvain Gilgen & Dokumentvorlage V1.0\\
        \hline
        Vorlage & 07.11.2019 & Sylvain Gilgen & Dokumentvorlage V1.0\\
        \hline
        Vorlage & 07.11.2019 & Sylvain Gilgen & Dokumentvorlage V1.0\\
        \hline
    \end{tabular}
    \caption{Änderungsprotokoll}
\end{table}
\chapter{Detailliertes Projektvorgehen}
\section{Projektvorgehen}
\subsection{Projektmethode}
\subsection{Szenario}
\subsection{Phasen}
\subsection{Module}
\section{Projektorganisation}
\subsection{Projektrollen}

\chapter{Risikoanalyse}
\begin{table}[H]
    \begin{tabular}{ |C{0.1\textwidth}|C{0.2\textwidth}|C{0.2\textwidth}|C{0.1\textwidth}|C{0.1\textwidth}|C{0.1\textwidth}|C{0.2\textwidth}| }
        \hline
        Nr & Riskiobeschreibung & Auswirkung & W & S & Risiko & Handlungsweise \\
        \hline 
        1 & Krankheit & Weniger Arbeitstage & 3 & 2-4 & \cellcolor{red}Hoch &\\
        \hline
        2 & Datenverslust &Datenverlus und Zeitverzug & W2 & S4 & \cellcolor{red}Hoch &\\
        \hline
        3 & Ausfall des Laptops & Zeistverzug & W3 & S2 & \cellcolor{yellow}Niedrig &\\
        \hline
    \end{tabular}
    \caption{Risikoanalyse}
\end{table}

\bigbreak
\begin{table}[H]
    \begin{tabular}{ |C{0.1\textwidth}|C{0.3\textwidth}|C{0.1\textwidth}|C{0.1\textwidth}|C{0.1\textwidth}|C{0.2\textwidth}| }
        \hline
        Risiko Nr & Massnahme &  W & S & Risiko & Handlungsweise \\
        \hline 
        1 & Keine Risken eingehen bei Kälte, immer genug Kleidung tragen 
        & W3 & S3 & \cellcolor{yellow}Mittel & Risikoakzeptanz \\
        \hline
        2 & Dateisicherungskonzept erstellen und dies strikt einhalten 
        & W2 & S2 & \cellcolor{green}Niedrig & Risikoakzeptanz \\
        \hline
        3 & Sorge tragen und ein backup Gerät bereitstellen 
        & W1 & S1 & \cellcolor{green}Niedrig & Risikoakzeptanz \\
        \hline
    \end{tabular}
    \caption{Massnahmen}
\end{table}
Schadensausmaß: \\
S1 = führt zu keiner Abwertung \\
S2 = geringe Abwertung bis 1.0 Notenpunkte \\
S3 = hohe Abwertung über 1,0 Notenpunkte \\
S4 = führt zu Nichtbestehen \\
\newline
Eintrittswahrscheinlichkeit: \\
W1 = unvorstellbar \\
W2 = unwahrscheinlich \\
W3 = eher vorstellbar \\
W4 = vorstellbar \\
W5= Eintreffen hoch \\

\section{Risikomatrix}


\begin{table}[H]
    \renewcommand{\arraystretch}{4}
    \begin{tabular}{*{6}{|L{0.16\textwidth}}}
        \hline
        W5 & \cellcolor{yellow} & \cellcolor{red} &\cellcolor{red} & \cellcolor{red} \\
        \hline 
        W4 & \cellcolor{yellow} & \cellcolor{yellow} & \cellcolor{red} & \cellcolor{red}  \\
        \hline
        W3 & \cellcolor{green} \tikz\draw[black,fill=gray] circle [radius=0.2] node {1}; & \cellcolor{yellow} \tikz\draw[black,fill=white] circle [radius=0.2] node {1};  & \cellcolor{yellow} & \cellcolor{red} \\
        \hline 
        W2 & \cellcolor{green} & \cellcolor{green} & \cellcolor{yellow} & \cellcolor{yellow} \\
        \hline
        W1 & \cellcolor{green} & \cellcolor{green} & \cellcolor{green} & \cellcolor{green} \\
        \hline
        & S1 & S2 & S3 & S4 \\
        \hline
    \end{tabular}
    \renewcommand{\arraystretch}{1}
    \caption{Risikomatrix}
\end{table}

\begin{table}[H]
    \begin{tabular}{|L{0.2\textwidth}|L{0.7\textwidth}|}
        \hline
        \multicolumn{2}{|l|}{Legende} \\
        \hline
        \tikz\draw[black,fill=white] circle [radius=0.2] node {1}; & Risiko ohne Massnahme \\
        \hline
        \tikz\draw[black,fill=gray] circle [radius=0.2] node {1}; & Risiko nach Massnahme \\
        \hline
    \end{tabular}
    \caption{Riskiomatrix Legende}
\end{table}

\subsection{Kurze Stellungnahme zu den Risiken}

\chapter{Zeitplan}
\section{Zeitplan Tabelle}
\section{Meilensteine}

\chapter{Arbeitsjournal}
\section{Tag 1 1.1.1029}
\begin{table}[H]
    \begin{tabular}{|L{0.4\textwidth}|C{60pt}|C{60pt}|C{60pt}|}
        \hline
        \rowcolor{puzzleblue!50}Tätigkeiten & Beteiligte Personen & Aufwand Geplant(Std) & Aufwand Effektiv \\
        \hline
        arbeit 1 & 1 & 1 & 1 \\
        \hline
    \end{tabular}
    \caption{Tätigkeiten}
\end{table}
\subsection{Tagesablauf}
\lipsum[1-3]
\subsection{Hilfestellungen}
\subsection{Reflexion}
\subsubsection*{Was lief gut}

\subsubsection*{Was lief weniger gut}

\subsubsection*{Deine Erkenntnisse von heute}


\subsection{Nächste Schritte}

\chapter{Abschlussbericht}
\section{Vergleich Ist/Soll}
\section{Mittelbedarf}
\section{Realisierungsbericht}
\section{Testbericht}
\section{Fazit zum IPA (Projekt)}
\section{Persönliches Fazit}
\section{Schlussreflexion}

\chapter{Selbständigkeitserklärung und Rechtliches}
