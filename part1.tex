\part{Ablauf Organisation und Umfeld}
\chapter{Kurzfassung des IPA-Berichtes}
\section{Ausgangssituation}
\section{Umsetzung}
\section{Ergebnis}

\tableofcontents



\chapter{Aufgabenstellung}
\section{Titel der Arbeit}
\section{Thematik}
\section{Ausgangslage}
\section{Detaillierte Aufgabenstellung}
\section{Mittel und Methoden inklusive Projektmethode}
\section{Vorkenntnisse}
\section{Vorarbeiten}
\section{Neue Lerninhalte}
\section{Arbeiten in den letzten 6 Monaten}


\chapter{Standards}

\chapter{Schutzwertanalyse}
%Dateisicherung nicht mehr verlangt

\chapter{Organisation der IPA}
\section{Datensicherung der IPA}
\section{Änderungskontrolle, Prüfung, Genehmigung}
\begin{table}[h!]
    \begin{tabular}{|L{0.1\textwidth}|L{0.23\textwidth}|L{0.23\textwidth}|L{0.33\textwidth}|}
        \hline
        \rowcolor{puzzleblue}Version & Datum & Name & Beschreibung \\[11pt]
        \hline
        Vorlage & 07.11.2019 & Sylvain Gilgen & Dokumentvorlage V1.0\\
        \hline
        Vorlage & 07.11.2019 & Sylvain Gilgen & Dokumentvorlage V1.0\\
        \hline
        Vorlage & 07.11.2019 & Sylvain Gilgen & Dokumentvorlage V1.0\\
        \hline
    \end{tabular}
    \caption{Änderungsprotokoll}
\end{table}
\chapter{Detailliertes Projektvorgehen}
\section{Projektvorgehen}
\subsection{Projektmethode}
\subsection{Szenario}
\subsection{Phasen}
\subsection{Module}
\section{Projektorganisation}
\subsection{Projektrollen}

\input{risiko.tex}

\section{Risikomatrix}


\begin{table}[H]
    \renewcommand{\arraystretch}{4}
    \begin{tabular}{*{6}{|L{0.16\textwidth}}}
        \hline
        W5 & \cellcolor{yellow} & \cellcolor{red} &\cellcolor{red} & \cellcolor{red} \\
        \hline 
        W4 & \cellcolor{yellow} & \cellcolor{yellow} & \cellcolor{red} & \cellcolor{red}  \\
        \hline
        W3 & \cellcolor{green} \tikz\draw[black,fill=gray] circle [radius=0.2] node {1}; & \cellcolor{yellow} \tikz\draw[black,fill=white] circle [radius=0.2] node {1};  & \cellcolor{yellow} & \cellcolor{red} \\
        \hline 
        W2 & \cellcolor{green} & \cellcolor{green} & \cellcolor{yellow} & \cellcolor{yellow} \\
        \hline
        W1 & \cellcolor{green} & \cellcolor{green} & \cellcolor{green} & \cellcolor{green} \\
        \hline
        & S1 & S2 & S3 & S4 \\
        \hline
    \end{tabular}
    \renewcommand{\arraystretch}{1}
    \caption{Risikomatrix}
\end{table}

\begin{table}[H]
    \begin{tabular}{|L{0.2\textwidth}|L{0.7\textwidth}|}
        \hline
        \multicolumn{2}{|l|}{Legende} \\
        \hline
        \tikz\draw[black,fill=white] circle [radius=0.2] node {1}; & Risiko ohne Massnahme \\
        \hline
        \tikz\draw[black,fill=gray] circle [radius=0.2] node {1}; & Risiko nach Massnahme \\
        \hline
    \end{tabular}
    \caption{Riskiomatrix Legende}
\end{table}

\subsection{Kurze Stellungnahme zu den Risiken}

\chapter{Zeitplan}


\section{Erläuterung zum Zeitplan}
Der Zeitplan auf der nächsten Seite ist in eine Blockgrösse von zwei Stunden aufgeteilt. Blöcke von nur einer Stunde werden durch gleichzeitiges Arbeiten an zwei 2-Stundenblöcken dargestellt.
\newpage
\storeareas\timevalues
\KOMAoptions{paper=a3, paper=landscape, DIV=current}
\areaset
{\dimexpr\the\paperwidth-2cm\relax}% calculate requiered \textwidth
{\dimexpr\the\paperheight-5.5cm\relax}% calculate requiered \textheight
\recalctypearea


\newcommand{\soll}[3]{\ganttbar{#1}{#2}{#3}}
\newcommand{\ist}[2]{\ganttbar[bar top shift=-0.01,bar/.append style={bottom color=puzzleblue, top color=puzzleblue!50}]{}{#1}{#2}}
\begin{table}

    \begin{ganttchart}[
            title height=1,
            y unit title=9mm,
            y unit chart=4.2mm,
            x unit=4.4mm,
            hgrid={dotted, draw=none},
            vgrid={draw=none, dotted, draw=none, black},
            % BAR
            bar label node/.append style={yshift=-1.3mm},
            bar height=.7,
            bar top shift=.3,
            bar/.append style={top color=lightgray!50, bottom color=gray},
            % MILESTONE
            milestone/.append style={top color=red!50, bottom color=red},
            milestone height=0.4mm,
            milestone inline label node/.style={below=2mm, font=\scriptsize},
            milestone top shift=-0.15mm,
            %bar label node/.style={text width=4cm,align=right,anchor=east},
        ]{1}{80}
        \pgfcalendar{titlecal}{2020-04-17}{2020-04-30}{
            % Exclude weekends and mondays
            \ifdate{weekend}{}{{
                    \gantttitle{
                        \pgfcalendarweekdayshortname{\pgfcalendarcurrentweekday}
                        \pgfcalendarcurrentday.\pgfcalendarcurrentmonth
                    }{8}
                }
            }
        } \\

        \gantttitle{Initialisierung}{14}
        \gantttitle{Konzept}{16}
        \gantttitle{Realisierung}{8}
        \gantttitle{Abschluss}{10} \\
        \gantttitlelist{1,...,40}{2}\\
        %%%%%% Allgemeines
        \soll{Dokumentation / Journal}{8}{8}
        \soll{}{16}{16}
        \soll{}{24}{24}
        \soll{}{32}{32}
        \soll{}{40}{40}
        \soll{}{48}{48}
        \soll{}{56}{56}
        \soll{}{64}{64}
        \soll{}{72}{72}
        \soll{}{80}{80}\\
        \ist{8}{8}
        \ist{16}{16}
        \ist{24}{24}
        \ist{32}{32}
        \ist{40}{40}
        \ist{48}{48}
        \ist{56}{56}
        \ist{64}{64}
        \ist{72}{72}
        \ist{80}{80} \\
        \soll{Expertenbesuch}{26}{26}
        \soll{}{39}{39}\\
        \ist{26}{26}
        \ist{39}{39}\\
        %%%%%% DAY ONE 1-8
        \soll{Zeitplan}{1}{2} \\
        \ist{1}{2}\\
        \soll{Aufgabenstellung}{3}{3} \\
        \ist{3}{3}\\
        \soll{Organisation IPA}{4}{4} \\
        \ist{4}{4}\\
        \soll{Firmenstandarts}{5}{5} \\
        \ist{5}{5}\\
        \soll{Ist Zustand}{6}{8} \\
        \ist{6}{9}\\
        \soll{Risikoanalyse}{9}{10} \\
        \ist{10}{11}\\
        \soll{Einführung}{11}{11} \\
        \ist{12}{12} \\
        \soll{Analyse}{12}{14} \\
        \ist{13}{16} \\
        \soll{Variantenetnscheid}{15}{16} \\
        \ist{17}{18} 
        \ganttmilestone[inline]{M1}{16}\\
        %%%%%% DAY TWO 9-16
        \soll{Aufgabe 1}{17}{20}\\
        \ist{19}{22} \\
        \soll{Aufgabe 2}{21}{24}\\
        \ist{21}{24}\\
        \\
        \soll{Aufgabe 3}{25}{28}\\
        \ist{25}{27} \\
        \soll{Aufgabe 4}{29}{31}\\
        \ist{28}{30} \\
        \soll{Aufgabe 5}{29}{32}\\
        \ist{31}{32} 
        \ganttmilestone[inline]{M2}{32} \\
        %%%%%% DAY THREE 17-24
        \soll{Aufgabe 6}{33}{38}\\
        \ist{33}{38} 
        \ganttmilestone[inline]{M3}{38} \\
        \soll{Aufgabe 7}{39}{40}\\
        \ist{39}{40} 
        \ganttmilestone[inline]{M4}{40} \\
        %%%%%% DAY FOUR 13-16
        \soll{Abschlussbericht}{41}{42}\\
        \ist{41}{42} \\
        \soll{Dokument formattieren/korigieren}{43}{48}\\
        \ist{43}{48} 
        \ganttmilestone[inline]{M5}{48}
        %%%%%% EXPERTENBESUCHE
    \end{ganttchart}
    \caption{Zeitplan}
\end{table}

{\addtolength{\leftskip}{5.1cm}
\begin{table}[!h]
    \begin{tabular}{L{0.05\textwidth}L{0.08\textwidth}L{0.05\textwidth}L{0.2\textwidth}}
        \textbf{Legende} &      & \multicolumn{2}{l}{\textbf{Meilenstene}}                                 \\
        Grau:            & soll & M1:                                      & Initialisierung abgeschlossen \\
        Blau:            & ist  & M2:                                      & Konzeptphase abgeschlossen    \\
                         &      & M3:                                      & POC erstellt    \\
                         &      & M4:                                      & Realisierung abgeschlossen \\
                         &      & M5:                                      & Projekt abgeschlossen \\
    \end{tabular}
    \caption{Legende und Meilensteine vom Zeitplan}
\end{table}
}

\restoregeometry
\timevalues
\newpage

\section{Meilensteine}

\chapter{Arbeitsjournal}
\section{Tag 1 1.1.1029}
\begin{table}[H]
    \begin{tabular}{|L{0.4\textwidth}|C{60pt}|C{60pt}|C{60pt}|}
        \hline
        \rowcolor{puzzleblue!50}Tätigkeiten & Beteiligte Personen & Aufwand Geplant(Std) & Aufwand Effektiv \\
        \hline
        arbeit 1 & 1 & 1 & 1 \\
        \hline
        arbeit 1 & 1 & 1 & 1 \\
        \hline
        arbeit 1 & 1 & 1 & 1 \\
        \hline
        arbeit 1 & 1 & 1 & 1 \\
        \hline
    \end{tabular}
    \caption{Tätigkeiten}
\end{table}
\subsection{Tagesablauf}
\lipsum[1-3]
\subsection{Hilfestellungen}
\subsection{Reflexion}
\subsubsection*{Was lief gut}

\subsubsection*{Was lief weniger gut}

\subsubsection*{Deine Erkenntnisse von heute}


\subsection{Nächste Schritte}

\chapter{Abschlussbericht}
\section{Vergleich Ist/Soll}
\section{Mittelbedarf}
\section{Realisierungsbericht}
\section{Testbericht}
\section{Fazit zum IPA (Projekt)}
\section{Persönliches Fazit}
\section{Schlussreflexion}

\chapter{Selbständigkeitserklärung und Rechtliches}
