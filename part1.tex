\part{Ablauf Organisation und Umfeld}
\chapter{Kurzfassung des IPA-Berichtes}
\section{Ausgangssituation}
\section{Umsetzung}
\section{Ergebnis}

\tableofcontents



\chapter{Aufgabenstellung}
\section{Titel der Arbeit}
\section{Thematik}
\section{Ausgangslage}
\section{Detaillierte Aufgabenstellung}
\section{Mittel und Methoden inklusive Projektmethode}
\section{Vorkenntnisse}
\section{Vorarbeiten}
\section{Neue Lerninhalte}
\section{Arbeiten in den letzten 6 Monaten}


\chapter{Standards}

\chapter{Schutzwertanalyse}
%Dateisicherung nicht mehr verlangt

\chapter{Organisation der IPA}
\section{Datensicherung der IPA}
\section{Änderungskontrolle, Prüfung, Genehmigung}
\begin{table}[h!]
    \begin{tabular}{|L{0.1\textwidth}|L{0.23\textwidth}|L{0.23\textwidth}|L{0.33\textwidth}|}
        \hline
        \rowcolor{puzzleblue}Version & Datum & Name & Beschreibung \\[11pt]
        \hline
        Vorlage & 07.11.2019 & Sylvain Gilgen & Dokumentvorlage V1.0\\
        \hline
        Vorlage & 07.11.2019 & Sylvain Gilgen & Dokumentvorlage V1.0\\
        \hline
        Vorlage & 07.11.2019 & Sylvain Gilgen & Dokumentvorlage V1.0\\
        \hline
    \end{tabular}
    \caption{Änderungsprotokoll}
\end{table}
\chapter{Detailliertes Projektvorgehen}
\section{Projektvorgehen}
\subsection{Projektmethode}
\subsection{Szenario}
\subsection{Phasen}
\subsection{Module}
\section{Projektorganisation}
\subsection{Projektrollen}

\newpage
\storeareas\riskvalues
\KOMAoptions{paper=a3, paper=landscape, DIV=current}
\areaset
  {\dimexpr\the\paperwidth-2cm\relax}% calculate requiered \textwidth
  {\dimexpr\the\paperheight-5.5cm\relax}% calculate requiered \textheight
\recalctypearea

\chapter{Risikoanalyse}
\begin{table}[H]
    \begin{tabular}{ |C{0.01\textwidth}|C{0.1\textwidth}|C{0.1\textwidth}|C{0.02\textwidth}|C{0.02\textwidth}|C{0.1\textwidth}|C{0.1\textwidth}|C{0.05\textwidth}|C{0.05\textwidth}|C{0.1\textwidth}|C{0.1\textwidth}|C{0.1\textwidth}|C{0.2\textwidth}| }
        \hline
        Nr & Riskiobeschreibung & Auswirkung & W & S & Risiko & Handlungsweise & Massnahme &  W & S & Risiko & Handlungsweise \\
        \hline 
        1 & Krankheit & Weniger Arbeitstage & W3 & S2-4 & \cellcolor{red}Hoch & Keine Risken eingehen bei Kälte, immer genug Kleidung tragen 
        & W3 & S3 & \cellcolor{yellow}Mittel & Risikoakzeptanz &\\
        \hline
        2 & Datenverslust &Datenverlus und Zeitverzug & W2 & S4 & \cellcolor{red}Hoch &Dateisicherungskonzept erstellen und dies strikt einhalten 
        & W2 & S2 & \cellcolor{green}Niedrig & Risikoakzeptanz &\\
        \hline
        3 & Ausfall des Laptops & Zeistverzug & W3 & S2 & \cellcolor{yellow}Niedrig &Sorge tragen und ein backup Gerät bereitstellen 
        & W1 & S1 & \cellcolor{green}Niedrig & Risikoakzeptanz &\\
        \hline
    \end{tabular}
    \caption{Risikoanalyse}
\end{table}

Schadensausmaß: \\
S1 = führt zu keiner Abwertung \\
S2 = geringe Abwertung bis 1.0 Notenpunkte \\
S3 = hohe Abwertung über 1,0 Notenpunkte \\
S4 = führt zu Nichtbestehen \\
\newline
Eintrittswahrscheinlichkeit: \\
W1 = unvorstellbar \\
W2 = unwahrscheinlich \\
W3 = eher vorstellbar \\
W4 = vorstellbar \\
W5= Eintreffen hoch \\

\restoregeometry
\riskvalues
\newpage

\section{Risikomatrix}


\begin{table}[H]
    \renewcommand{\arraystretch}{4}
    \begin{tabular}{*{6}{|L{0.16\textwidth}}}
        \hline
        W5 & \cellcolor{yellow} & \cellcolor{red} &\cellcolor{red} & \cellcolor{red} \\
        \hline 
        W4 & \cellcolor{yellow} & \cellcolor{yellow} & \cellcolor{red} & \cellcolor{red}  \\
        \hline
        W3 & \cellcolor{green} \tikz\draw[black,fill=gray] circle [radius=0.2] node {1}; & \cellcolor{yellow} \tikz\draw[black,fill=white] circle [radius=0.2] node {1};  & \cellcolor{yellow} & \cellcolor{red} \\
        \hline 
        W2 & \cellcolor{green} & \cellcolor{green} & \cellcolor{yellow} & \cellcolor{yellow} \\
        \hline
        W1 & \cellcolor{green} & \cellcolor{green} & \cellcolor{green} & \cellcolor{green} \\
        \hline
        & S1 & S2 & S3 & S4 \\
        \hline
    \end{tabular}
    \renewcommand{\arraystretch}{1}
    \caption{Risikomatrix}
\end{table}

\begin{table}[H]
    \begin{tabular}{|L{0.2\textwidth}|L{0.7\textwidth}|}
        \hline
        \multicolumn{2}{|l|}{Legende} \\
        \hline
        \tikz\draw[black,fill=white] circle [radius=0.2] node {1}; & Risiko ohne Massnahme \\
        \hline
        \tikz\draw[black,fill=gray] circle [radius=0.2] node {1}; & Risiko nach Massnahme \\
        \hline
    \end{tabular}
    \caption{Riskiomatrix Legende}
\end{table}

\subsection{Kurze Stellungnahme zu den Risiken}

\chapter{Zeitplan}


\section{Erläuterung zum Zeitplan}
Der Zeitplan auf der nächsten Seite ist in eine Blockgrösse von zwei Stunden aufgeteilt. Blöcke von nur einer Stunde werden durch gleichzeitiges Arbeiten an zwei 2-Stundenblöcken dargestellt.
\newpage
\storeareas\timevalues
\KOMAoptions{paper=a3, paper=landscape, DIV=current}
\areaset
  {\dimexpr\the\paperwidth-2cm\relax}% calculate requiered \textwidth
  {\dimexpr\the\paperheight-5.5cm\relax}% calculate requiered \textheight
\recalctypearea


\newcommand{\soll}[3]{\ganttbar{#1}{#2}{#3}}
\newcommand{\ist}[2]{\ganttbar[bar top shift=-0.01,bar/.append style={bottom color=puzzleblue}]{}{#1}{#2}}

\begin{ganttchart}[
        title height=1,
        y unit title=9mm,
        y unit chart=8mm,
        x unit=4.4mm,
        hgrid={dotted, draw=none},
        vgrid={draw=none, dotted, draw=none, black},
        % BAR
        bar label node/.append style={yshift=-1.3mm},
        bar height=.7,
        bar top shift=.3,
        bar/.append style={top color=lightgray!50, bottom color=gray},
        % MILESTONE
        milestone/.append style={top color=white, bottom color=gray},
        milestone height=0.4mm,
        milestone inline label node/.style={below=2mm, font=\scriptsize},
        milestone top shift=-0.15mm,
        %bar label node/.style={text width=4cm,align=right,anchor=east},
    ]{1}{80}
    \pgfcalendar{titlecal}{2016-05-06}{2016-05-24}{
        % Exclude weekends and mondays
        \ifdate{weekend}{}{\ifdate{Monday}{}{
                \gantttitle{
                    \pgfcalendarweekdayshortname{\pgfcalendarcurrentweekday}
                    \pgfcalendarcurrentday.\pgfcalendarcurrentmonth
                }{8}
            }
        }
    } \\

    \gantttitle{Initialisierung}{12}
    \gantttitle{Konzept}{24}
    \gantttitle{Realisierung}{32}
    \gantttitle{Abschluss}{12} \\
    \gantttitlelist{1,...,40}{2}\\
    %%%%%% EXPERTENBESUCHE
    \soll{Expertenbesuche}{14}{14}
    \soll{}{54}{54} \\
    \ist{14}{14}
    \ist{54}{54} \\
    %%%%%% JOURNAL & DOKUMENTATION
    \soll{Dokumentation / Journal}{4}{12}
    \soll{}{16}{36}
    \soll{}{46}{48}
    \soll{}{56}{56}
    \soll{}{64}{68}
    \soll{}{72}{72} \\
    \ist{4}{12}
    \ist{16}{18}
    \ist{20}{20}
    \ist{23}{24}
    \ist{28}{28}
    \ist{32}{34}
    \ist{36}{36}
    \ganttmilestone[inline]{M1}{12}
    \ganttmilestone[inline]{M2}{36} \\
    %%%%%% DAY ONE 1-4
    \soll{Zeitplan}{1}{1} \\
    \ist{1}{1} \\
    %%%%%% DAY TWO 5-8
    %%%%%% DAY THREE 9-12
    %%%%%% DAY FOUR 13-16
    %%%%%% DAY FIVE 17-20
    %%%%%% DAY SIX 21-24
    \soll{Scripts schreiben}{19}{22} \\
    \ist{19}{22} \\
    %%%%%% DAY SEVEN 25-28
    \soll{Product}{25}{26} \\
    \ist{25}{26} \\
    \soll{Product2}{27}{27} \\
    \ist{27}{27} \\
    %%%%%% DAY EIGHT 29-32
    \soll{Product3}{29}{29} \\
    \ist{29}{30} \\
    \soll{Testing}{30}{31} \\
    \ist{30}{31}
    \ganttmilestone[inline, milestone top shift=-0.4mm]{M3}{34} \\
    %%%%%% DAY NINE 33-36
    \soll{Reserve}{35}{35} \\
    \ist{35}{35} \\
    %%%%%% DAY TEN 37-40
    \soll{Abschluss}{37}{40} \\
    \ist{37}{40}
    \ganttmilestone[inline]{M4}{39}
\end{ganttchart}

{\addtolength{\leftskip}{5.1cm}
\textbf{Legende}\\
Grau: soll \\
Blau: ist

\textbf{Meilensteine}\\
M1: Initialisierung abgeschlossen\\
M2: Konzeptphase abgeschlossen\\
M3: Realisierung abgeschlossen\\
M4: Projekt abgeschlossen

}

\restoregeometry
\timevalues
\newpage

\section{Meilensteine}

\chapter{Arbeitsjournal}
\section{Tag 1 1.1.1029}
\begin{table}[H]
    \begin{tabular}{|L{0.4\textwidth}|C{60pt}|C{60pt}|C{60pt}|}
        \hline
        \rowcolor{puzzleblue!50}Tätigkeiten & Beteiligte Personen & Aufwand Geplant(Std) & Aufwand Effektiv \\
        \hline
        arbeit 1 & 1 & 1 & 1 \\
        \hline
        arbeit 1 & 1 & 1 & 1 \\
        \hline
        arbeit 1 & 1 & 1 & 1 \\
        \hline
        arbeit 1 & 1 & 1 & 1 \\
        \hline
    \end{tabular}
    \caption{Tätigkeiten}
\end{table}
\subsection{Tagesablauf}
\lipsum[1-3]
\subsection{Hilfestellungen}
\subsection{Reflexion}
\subsubsection*{Was lief gut}

\subsubsection*{Was lief weniger gut}

\subsubsection*{Deine Erkenntnisse von heute}


\subsection{Nächste Schritte}

\chapter{Abschlussbericht}
\section{Vergleich Ist/Soll}
\section{Mittelbedarf}
\section{Realisierungsbericht}
\section{Testbericht}
\section{Fazit zum IPA (Projekt)}
\section{Persönliches Fazit}
\section{Schlussreflexion}

\chapter{Selbständigkeitserklärung und Rechtliches}
