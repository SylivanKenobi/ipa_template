\documentclass{report}

\usepackage[utf8]{inputenc}
\usepackage{lmodern}
\usepackage{amsmath}
\usepackage[ngerman]{babel}
\usepackage{graphicx}
\usepackage{fancyhdr}
\usepackage{roboto}
\usepackage{sectsty}
\usepackage{lastpage}
\usepackage{tabularx}
\usepackage[table]{xcolor}
\usepackage{tikz}

\textwidth=400pt
\marginparwidth=90pt
\allsectionsfont{\sffamily}
\fancypagestyle{plain}{}
\pagestyle{fancy}
\setlength\headheight{33pt}
\graphicspath{{bilder/}}
\fancyhf{} 
\lhead{\includegraphics[width=3cm]{puzzle_tagline_blau}}  
\rfoot{Seite \thepage \hspace{1pt} von \pageref{LastPage}}
\title{Title}
\author{Sylvain}

\begin{document}

\maketitle

\tableofcontents

\part{Ablauf Organisation und Umfeld}

\chapter{Aufgabenstellung}
\section{Titel der Arbeit}
\section{Thematik}
\section{Ausgangslage}
\section{Detaillierte Aufgabenstellung}
\section{Mittel und Methoden inklusive Projektmethode}
\section{Vorkenntnisse}
\section{Vorarbeiten}
\section{Neue LerninhalteArbeiten in den letzten 6 Monaten}


\chapter{Standards}

\chapter{ISDS}

\chapter{Organisation der IPA}
\section{Datensicherung der IPA}
\section{Änderungskontrolle, Prüfung, Genehmigung}

\chapter{Detailliertes Projektvorgehen}
\subsection{Projektmethode}
\subsection{Szenario}
\subsection{Phasen}
\subsection{Module}

\section{Risikoanalyse}

\begin{tabular}{ |c|c|c|c|c|c| }
    \hline
    Nr & Riskiobeschreibung & Auswirkung & W & S & Riskio \\
    \hline 
    1 & Krankheit & Weniger Arbeitstage & 3 & 2-4 & \cellcolor{red}Hoch \\
    \hline
    2 & Datenverslust & Weniger Arbeitstage & 3 & 2-4 & \cellcolor{red}Hoch \\
    \hline
    1 & Krankheit & Weniger Arbeitstage & 3 & 2-4 & \cellcolor{red}Hoch \\
    \hline
    1 & Krankheit & Weniger Arbeitstage & 3 & 2-4 & \cellcolor{red}Hoch \\
    \hline
    1 & Krankheit & Weniger Arbeitstage & 3 & 2-4 & \cellcolor{red}Hoch \\
    \hline
    1 & Krankheit & Weniger Arbeitstage & 3 & 2-4 & \cellcolor{red}Hoch \\
    \hline
    1 & Krankheit & Weniger Arbeitstage & 3 & 2-4 & \cellcolor{red}Hoch \\
    \hline
\end{tabular}
\newline
\bigbreak
\begin{flushleft}
\begin{tabular}{ |c|c|c|c|c|c| }
    \hline
    Risiko Nr & Massnahme &  W & S & Riskio & Handlungsweise \\
    \hline 
    1 & Krankheit & Weniger Arbeitstage & 3 & 2-4 & \cellcolor{red}Hoch \\
    \hline
    1 & Krankheit & Weniger Arbeitstage & 3 & 2-4 & \cellcolor{red}Hoch \\
    \hline
    1 & Krankheit & Weniger Arbeitstage & 3 & 2-4 & \cellcolor{red}Hoch \\
    \hline
    1 & Krankheit & Weniger Arbeitstage & 3 & 2-4 & \cellcolor{red}Hoch \\
    \hline
    1 & Krankheit & Weniger Arbeitstage & 3 & 2-4 & \cellcolor{red}Hoch \\
    \hline
    1 & Krankheit & Weniger Arbeitstage & 3 & 2-4 & \cellcolor{red}Hoch \\
    \hline
    1 & Krankheit & Weniger Arbeitstage & 3 & 2-4 & \cellcolor{red}Hoch \\
    \hline
\end{tabular}
    
\end{flushleft}
\section{Risikomatrix}


\begin{center}
\renewcommand{\arraystretch}{2}
\begin{tabularx}{200pt}{ |X|X|X|X| }
    \hline
    \cellcolor{yellow} & \cellcolor{red} &\cellcolor{red} & \cellcolor{red} \\
    \hline 
    \cellcolor{yellow} & \cellcolor{yellow} & \cellcolor{red} & \cellcolor{red}  \\
    \hline
    \cellcolor{green} & \cellcolor{yellow} \tikz\draw[black,fill=red] circle [radius=0.2] node {1};  & \cellcolor{yellow} & \cellcolor{red} \\
    \hline 
    \cellcolor{green} & \cellcolor{green} & \cellcolor{yellow} & \cellcolor{yellow} \\
    \hline
    \cellcolor{green} & \cellcolor{green} & \cellcolor{green} & \cellcolor{green} \\
    \hline
\end{tabularx}
\renewcommand{\arraystretch}{1}
\end{center}

\subsection{Kurze Stellungnahme zu den Risiken}

\chapter{Arbeitsjournal}

\chapter{Abschlussbericht}
\section{Vergleich Ist/Soll}
\section{Mittelbedarf}
\section{Realisierungsbericht}
\section{Testbericht}
\section{Fazit zum IPA (Projekt)}
\section{Persönliches Fazit}
\section{Schlussreflexion}

\chapter{Selbständigkeitserklärung und Rechtliches}

\part{Projektdokumentation}

\chapter{Analyseteil (Initialisierung nach Hernes IPA)}
\section{Studie; Ist-Zustand}
\section{Persönliche Vorgehensziele}
\section{Projektziele}
\section{Varianten}
\section{Anforderungen}
\subsection{Funktionale Anforderungen}
\subsection{Nicht funktionale Anforderungen}
\section{Variantenvorstellung}
\subsection{Variante 1}
\section{Variantenentscheid}
\subsection{Begründung}

\chapter{Entwurf}
\section{Konzept entwickeln}
\section{BSP Testkonzept und Protokoll}

\chapter{Umsetzung}
\section{System erstellen}

\part{Formaler Teil 2}
\chapter{Abbildungsverzeichnis}

\chapter{Tabellenverzeichnis}

\chapter{Literatur und Quellenverzeichnis}
\section{Internet Quelle: (Achtung löschen nur BSP.)}
\section{Literatur Quelle: (Achtung löschen nur BSP.)}

\chapter{Verwendete Abkürzungen}

\chapter{Glossar}

\chapter{Anhänge}
\end{document}