\documentclass{report}

\usepackage[utf8]{inputenc}
\usepackage{lmodern}
\usepackage{amsmath}
\usepackage[ngerman]{babel}
\usepackage{graphicx}
\usepackage{fancyhdr}
\usepackage{roboto}
\usepackage{sectsty}
\usepackage{lastpage}
\usepackage{tabularx}
\usepackage[table]{xcolor}
\usepackage{tikz}
\usepackage{array}
\usepackage{lipsum}
\usepackage{multirow}
\usepackage{float}

\makeatletter
\renewcommand\listoftables{%
    \section{\listtablename}%
    \@mkboth{\MakeUppercase\listtablename}%
        {\MakeUppercase\listtablename}%
    \@starttoc{lot}%
}
\renewcommand\listoffigures{%
    \section{\listfigurename}%
    \@mkboth{\MakeUppercase\listfigurename}%
        {\MakeUppercase\listfigurename}%
    \@starttoc{lof}%
}
\makeatother
\newcolumntype{L}[1]{>{\raggedright\let\newline\\\arraybackslash\hspace{0pt}}m{#1}}
\newcolumntype{C}[1]{>{\centering\let\newline\\\arraybackslash\hspace{0pt}}m{#1}}
\newcolumntype{R}[1]{>{\raggedleft\let\newline\\\arraybackslash\hspace{0pt}}m{#1}}

\definecolor{puzzleblue}{RGB}{30,90,150}
\textwidth=400pt
\oddsidemargin=47pt
\allsectionsfont{\sffamily}
\fancypagestyle{plain}{}
\pagestyle{fancy}
\setlength\headheight{33pt}
\graphicspath{{bilder/}}
\fancyhf{} 
\lhead{\includegraphics[width=3cm]{puzzle_tagline_blau}}  
\rfoot{Seite \thepage \hspace{1pt} von \pageref{LastPage}}
\author{Sylvain}
\title{IPA}
\lfoot{Sylvain Gilgen}

\begin{document}
\begin{titlepage}
  \Huge IPA Sylvain Gilgen Puzzle ITC\normalsize
\bigbreak
\begin{table}[h!]
    \begin{tabular}{|R{0.2\textwidth}|L{0.7\textwidth}|}
        \hline
        \rowcolor{puzzleblue} \multicolumn{2}{|l|}{\textbf{IPA-Daten}} \\[12pt]
        \hline
        Firma & Puzzle ITC \\
        \hline
        \rowcolor{puzzleblue!30}Abteilung & BBT \\
        \hline
        Autor & Sylvain Gilgen \\
        \hline
        \rowcolor{puzzleblue!30}Ausgabedatum & dd.mm.yyyy \\
        \hline
        Projektvorgehen & Hermes \\
        \hline
        \rowcolor{puzzleblue!30}Version & 0.09 \\
        \hline
      \end{tabular}
\end{table}
\begin{table}[!h]
    \caption{startseite}
    \begin{tabular}{|L{0.4\textwidth}|L{0.5\textwidth}|}
        \hline
        \rowcolor{puzzleblue} \multicolumn{2}{|l|}{\textbf{Beteiligte Personen}} \\[12pt]
        \hline
        \textbf{In der Genehmigung} & \\
        \hline
        \rowcolor{puzzleblue!30}Valid Experte & Vorname Nachname \\
        \hline
        \textbf{In der Durchführung} & \\
        \hline
        \rowcolor{puzzleblue!30}Verantwortliche Fachkraft & Vorname Nachname, Rolle \\
        \hline
        Hauptexperte & Vorname Nachname \\
        \hline
        \rowcolor{puzzleblue!30}Zweit Experte & Vorname Nachname \\
        \hline
        \textbf{Zur Information, Kenntnis} & \\
        \hline
        \rowcolor{puzzleblue!30}Berufsblidner & Vorname Nachname, Rolle \\
        \hline
    \end{tabular}
\end{table}
\end{titlepage}

\part{Ablauf Organisation und Umfeld}
\chapter{Kurzfassung des IPA-Berichtes}
\section{Ausgangssituation}
\section{Umsetzung}
\section{Ergebnis}

\tableofcontents



\chapter{Aufgabenstellung}
\section{Titel der Arbeit}
\section{Thematik}
\section{Ausgangslage}
\section{Detaillierte Aufgabenstellung}
\section{Mittel und Methoden inklusive Projektmethode}
\section{Vorkenntnisse}
\section{Vorarbeiten}
\section{Neue Lerninhalte}
\section{Arbeiten in den letzten 6 Monaten}


\chapter{Standards}

\chapter{Schutzwertanalyse}
%Dateisicherung nicht mehr verlangt

\chapter{Organisation der IPA}
\section{Datensicherung der IPA}
\section{Änderungskontrolle, Prüfung, Genehmigung}
\begin{table}[h!]
    \caption{Änderungsprotokoll}
    \begin{tabular}{|L{0.23\textwidth}|L{0.23\textwidth}|L{0.23\textwidth}|L{0.23\textwidth}|}
        \hline
        Version & Datum & Name & Beschreibung \\
        \hline
        Vorlage & 07.11.2019 & Sylvain Gilgen & Dokumentvorlage V1.0\\
        \hline
    \end{tabular}
\end{table}
\chapter{Detailliertes Projektvorgehen}
\section{Projektvorgehen}
\subsection{Projektmethode}
\subsection{Szenario}
\subsection{Phasen}
\subsection{Module}
\section{Projektorganisation}
\subsection{Projektrollen}

\chapter{Risikoanalyse}
\begin{tabular}{ |C{0.1\textwidth}|C{0.2\textwidth}|C{0.2\textwidth}|C{0.1\textwidth}|C{0.1\textwidth}|C{0.2\textwidth}| }
    \hline
    Nr & Riskiobeschreibung & Auswirkung & W & S & Risiko \\
    \hline 
    1 & Krankheit & Weniger Arbeitstage & 3 & 2-4 & \cellcolor{red}Hoch \\
    \hline
    2 & Datenverslust &Datenverlus und Zeitverzug & W2 & S4 & \cellcolor{red}Hoch \\
    \hline
    3 & Ausfall des Laptops & Zeistverzug & W3 & S2 & \cellcolor{yellow}Niedrig \\
    \hline
\end{tabular}
\newline
\bigbreak
\begin{flushleft}
    \begin{tabular}{ |C{0.1\textwidth}|C{0.3\textwidth}|C{0.1\textwidth}|C{0.1\textwidth}|C{0.1\textwidth}|C{0.2\textwidth}| }
        \hline
        Risiko Nr & Massnahme &  W & S & Risiko & Handlungsweise \\
        \hline 
        1 & Keine Risken eingehen bei Kälte, immer genug Kleidung tragen 
        & W3 & S3 & \cellcolor{yellow}Mittel & Risikoakzeptanz \\
        \hline
        2 & Dateisicherungskonzept erstellen und dies strikt einhalten 
        & W2 & S2 & \cellcolor{green}Niedrig & Risikoakzeptanz \\
        \hline
        3 & Sorge tragen und ein backup Gerät bereitstellen 
        & W1 & S1 & \cellcolor{green}Niedrig & Risikoakzeptanz \\
        \hline
    \end{tabular}
\end{flushleft}
Schadensausmaß: \\
S1 = führt zu keiner Abwertung \\
S2 = geringe Abwertung bis 1.0 Notenpunkte \\
S3 = hohe Abwertung über 1,0 Notenpunkte \\
S4 = führt zu Nichtbestehen \\
\newline
Eintrittswahrscheinlichkeit: \\
W1 = unvorstellbar \\
W2 = unwahrscheinlich \\
W3 = eher vorstellbar \\
W4 = vorstellbar \\
W5= Eintreffen hoch \\
\section{Risikomatrix}


\begin{center}
\renewcommand{\arraystretch}{4}
\begin{tabularx}{0.9\textwidth}{ |X|X|X|X|X| }
    \hline
    W5 & \cellcolor{yellow} & \cellcolor{red} &\cellcolor{red} & \cellcolor{red} \\
    \hline 
    W4 & \cellcolor{yellow} & \cellcolor{yellow} & \cellcolor{red} & \cellcolor{red}  \\
    \hline
    W3 & \cellcolor{green} & \cellcolor{yellow} \tikz\draw[black,fill=red] circle [radius=0.2] node {1};  & \cellcolor{yellow} & \cellcolor{red} \\
    \hline 
    W2 & \cellcolor{green} & \cellcolor{green} & \cellcolor{yellow} & \cellcolor{yellow} \\
    \hline
    W1 & \cellcolor{green} & \cellcolor{green} & \cellcolor{green} & \cellcolor{green} \\
    \hline
     & S1 & S2 & S3 & S4 \\
    \hline
\end{tabularx}
\renewcommand{\arraystretch}{1}
\end{center}

\subsection{Kurze Stellungnahme zu den Risiken}

\chapter{Zeitplan}
\section{Zeitplan Tabelle}
\section{Meilensteine}

\chapter{Arbeitsjournal}
\section{Tag 1 1.1.1029}
\begin{tabular}{|L{0.4\textwidth}|C{60pt}|C{60pt}|C{60pt}|}
    \hline
    \rowcolor{puzzleblue!50}Tätigkeiten & Beteiligte Personen & Aufwand Geplant(Std) & Aufwand Effektiv \\
    \hline
    arbeit 1 & 1 & 1 & 1 \\
    \hline
\end{tabular}
\subsection{Tagesablauf}
\lipsum[1-3]
\subsection{Hilfestellungen}
\subsection{Reflexion}
\subsubsection*{Was lief gut}

\subsubsection*{Was lief weniger gut}

\subsubsection*{Deine Erkenntnisse von heute}


\subsection{Nächste Schritte}

\chapter{Abschlussbericht}
\section{Vergleich Ist/Soll}
\section{Mittelbedarf}
\section{Realisierungsbericht}
\section{Testbericht}
\section{Fazit zum IPA (Projekt)}
\section{Persönliches Fazit}
\section{Schlussreflexion}

\chapter{Selbständigkeitserklärung und Rechtliches}

\part{Projektdokumentation}

\chapter{Initialisierung}
\section{Studie; Ist-Zustand}
\section{Persönliche Vorgehensziele}
\section{Projektziele}
\section{Varianten}
\section{Anforderungen}
\subsection{Funktionale Anforderungen}
\subsection{Nicht funktionale Anforderungen}
\section{Variantenvorstellung}
\subsection{Variante 1}
\lipsum[1-3]
\section{Variantenentscheid}
\begin{table}[H]
    \begin{tabular}{|L{0.2\textwidth}|L{0.13\textwidth}|L{0.13\textwidth}|L{0.13\textwidth}|L{0.13\textwidth}|L{0.13\textwidth}|}
        \hline
        \multirow{2}*{Kriterien} & \multirow{2}*{Gewicht} & \multicolumn{2}{|l|}{Lösung 1 } & \multicolumn{2}{|l|}{Lösung 2} \\
        \cline{3-6}
        & & Bewertung & Gesamt & Bewertung & Gesamt \\  
        \hline
        \rowcolor{puzzleblue!25}Erfüllung der Anforderungen & 25\% & 3 & 0,75 & 1 & 0,25 \\
        \hline
        Erfüllung der Anforderungen & 25\% & 3 & 0,75 & 1 & 0,25 \\
        \hline
        \rowcolor{puzzleblue!25}Erfüllung der Anforderungen & 25\% & 3 & 0,75 & 1 & 0,25 \\
        \hline
        Erfüllung der Anforderungen & 25\% & 3 & 0,75 & 1 & 0,25 \\
        \hline
        \rowcolor{puzzleblue!25}Erfüllung der Anforderungen & 25\% & 3 & 0,75 & 1 & 0,25 \\
        \hline
        \rowcolor{puzzleblue}\textbf{Gesamt} & \textbf{100\%} & & \textbf{0,75} & & \textbf{0.25} \\[12pt]
        \hline
    \end{tabular}
\end{table}   
LEGENDE 
\section{Begründung}

\chapter{Entwurf}
\section{Konzept entwickeln}
\section{BSP Testkonzept und Protokoll}
\subsection{Testinfrastruktur}
\subsection{Testziele}
\subsection{Testrahmen}
\subsection{Testvorgehen}
\subsection{Testmethode}
\subsection{Black-Box Tests}
\subsection{White-Box Tests}
\subsection{Testpersonen}
\subsection{Testverfahren 1}
\begin{tabular}{|l|l|}
    \hline
    \rowcolor{puzzleblue} \multicolumn{2}{|l|}{Testfall Nr. 1}  \\[10pt]
    \hline
    \textbf{Spezfikation} & \textbf{Beschreibung} \\
    \hline
    \rowcolor{puzzleblue!25}\textbf{Testname} & Rspec \\
    \hline
    \textbf{Testart} & Automatisiert \\
    \hline
    \rowcolor{puzzleblue!25}\textbf{Voraussetzungen} & Keine \\
    \hline
    \textbf{Vorbereitung} & Datenbank mit Testdaten starten \\
    \hline
    \rowcolor{puzzleblue!25}\textbf{Ablauf} & 1. Im Terminal rake spec eingeben \\
    \hline
    \textbf{Erwartetes Resultata} & Keine Fehler \\
    \hline
    \rowcolor{puzzleblue!25}\textbf{Erreichtes Resultat} & Keine Fehler \\
    \hline
\end{tabular}
\chapter{Umsetzung}
\section{System erstellen}

\part{Formaler Teil 2}

\chapter{Verzeichnisse}
\listoftables
\listoffigures
\begin{thebibliography}{9}
    \bibitem[Frank 04]{Kurs1} \emph{Erste Schritte mit \LaTeX},
    Sascha Frank 2004
    \bibitem[Frank 05]{kurz1} \emph{Kurzdokumentation zu Kurs 1}
    Sascha Frank 2005 
\end{thebibliography}
\addcontentsline{toc}{section}{Literaturverzeichnis}

\chapter{Verwendete Abkürzungen}
\begin{table}[h!]
    \begin{tabular}{|L{0.3\textwidth}|L{0.6\textwidth}|}
        \hline
        \rowcolor{puzzleblue!40} Abkürzung & Bedeutung \\
        \hline
    \end{tabular}
\end{table}

\chapter{Glossar}
\begin{table}[h!]
    \begin{tabular}{|L{0.3\textwidth}|L{0.6\textwidth}|}
        \hline
        \rowcolor{puzzleblue!40} Begriff & Bedeutung \\[12pt]
        \hline
    \end{tabular}
\end{table}
\chapter{Anhänge}
\section{Sitzungsprotokolle}
\subsection{Sitzung 1}
\subsection{Sitzung 2}
\end{document}