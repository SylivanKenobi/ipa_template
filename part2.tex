\part{Projektdokumentation}

\chapter{Initialisierung}
\section{Studie; Ist-Zustand}
\section{Persönliche Vorgehensziele}
\section{Projektziele}
\section{Varianten}
\section{Anforderungen}
\subsection{Funktionale Anforderungen}
\subsection{Nicht funktionale Anforderungen}
\section{Variantenvorstellung}
\subsection{Variante 1}
\lipsum[1-3]
\section{Variantenentscheid}
\begin{table}[H]
    \begin{tabular}{|L{0.2\textwidth}|L{0.13\textwidth}|L{0.13\textwidth}|L{0.13\textwidth}|L{0.13\textwidth}|L{0.13\textwidth}|}
        \hline
        \multirow{2}*{Kriterien} & \multirow{2}*{Gewicht} & \multicolumn{2}{|l|}{Lösung 1 } & \multicolumn{2}{|l|}{Lösung 2} \\
        \cline{3-6}
        & & Bewertung & Gesamt & Bewertung & Gesamt \\  
        \hline
        \rowcolor{puzzleblue!25}Erfüllung der Anforderungen & 25\% & 3 & 0,75 & 1 & 0,25 \\
        \hline
        Erfüllung der Anforderungen & 25\% & 3 & 0,75 & 1 & 0,25 \\
        \hline
        \rowcolor{puzzleblue!25}Erfüllung der Anforderungen & 25\% & 3 & 0,75 & 1 & 0,25 \\
        \hline
        Erfüllung der Anforderungen & 25\% & 3 & 0,75 & 1 & 0,25 \\
        \hline
        \rowcolor{puzzleblue!25}Erfüllung der Anforderungen & 25\% & 3 & 0,75 & 1 & 0,25 \\
        \hline
        \rowcolor{puzzleblue}\textbf{Gesamt} & \textbf{100\%} & & \textbf{0,75} & & \textbf{0.25} \\[12pt]
        \hline
    \end{tabular}
    \caption{Variantenentscheid}
\end{table}   
LEGENDE 
\section{Begründung}

\chapter{Entwurf}
\section{Konzept entwickeln}
\section{BSP Testkonzept und Protokoll}
\subsection{Testinfrastruktur}
\subsection{Testziele}
\subsection{Testrahmen}
\subsection{Testvorgehen}
\subsection{Testmethode}
\subsection{Black-Box Tests}
\subsection{White-Box Tests}
\subsection{Testpersonen}
\subsection{Testverfahren 1}
\begin{table}[H]
    \begin{tabular}{|l|l|}
        \hline
        \rowcolor{puzzleblue} \multicolumn{2}{|l|}{Testfall Nr. 1}  \\[10pt]
        \hline
        \textbf{Spezfikation} & \textbf{Beschreibung} \\
        \hline
        \rowcolor{puzzleblue!25}\textbf{Testname} & Rspec \\
        \hline
        \textbf{Testart} & Automatisiert \\
        \hline
        \rowcolor{puzzleblue!25}\textbf{Voraussetzungen} & Keine \\
        \hline
        \textbf{Vorbereitung} & Datenbank mit Testdaten starten \\
        \hline
        \rowcolor{puzzleblue!25}\textbf{Ablauf} & 1. Im Terminal rake spec eingeben \\
        \hline
        \textbf{Erwartetes Resultata} & Keine Fehler \\
        \hline
        \rowcolor{puzzleblue!25}\textbf{Erreichtes Resultat} & Keine Fehler \\
        \hline
    \end{tabular}
    \caption{Testfall 1}
\end{table}
\chapter{Umsetzung}
\section{System erstellen}
